\documentclass[a4paper,12pt]{article}
\usepackage{amsmath, amssymb}
\usepackage{graphicx}
\usepackage{hyperref}
\usepackage{arydshln} 

\title{Zero Sum Game to Linear Program}
\author{Andrew Spears}
\date{\today}

\begin{document}

\maketitle

Consider a normal-form zero-sum game with payoff matrix \( A \), where player 1 (P1) chooses a mixed strategy represented by the vector \( x \), and player 2 (P2) chooses a mixed strategy \( y \).

The expected utility for player 1 is given by:

\[
v = x^T A y
\]

where \( v \) represents the value of the game. It turns out that it makes more sense to frame $v$ as an independent decision variable.

\subsection*{Trivial Constraints}

The strategies \( x \) and \( y \) must satisfy the standard probability constraints:

\[
\sum_i x_i = 1, \quad 0 \leq x_i \quad \forall i
\]

\[
\sum_j y_j = 1, \quad 0 \leq y_j \quad \forall j
\]



\subsection*{Equilibrium Constraints}

A necessary condition for Nash equilibrium is that player 2 should be indifferent between all actions they play with positive probability. This is equivalent to requiring that the mixed strategy \( y \) should do at least as well for player 2 as any pure strategy, \underline{against the specific strategy \( x \)} (or rather, that the value $v$ for player 1 cannot be decreased by deviating $y$ to any pure strategy):

\[
v \leq x^T A e_j, \quad \forall j
\]

which can be rewritten as:

\[
v \cdot \mathbf{1} \leq x^T A I = x^T A
\]
\[ x^T A \geq v \cdot \mathbf{1} \]

where $\mathbf{1}$ is a vector of all 1s.
Notice that $y$ and the indices $j$ do not appear in this constraint.

All in all, we get the linear program:

\begin{equation}
\boxed{
\begin{array}{rl}
    \text{Maximize } & v \\
    \text{Subject to} & \\
    & x^T A \geq v \cdot \mathbf{1} \\
    & \sum_i x_i = 1 \\
    & 0 \leq x_i \quad \forall i \\
\end{array}
}
\end{equation}

The solution to this linear program will give the value \( v \) of the game and the Nash equilibrium strategy \( x \) for player 1\footnote{Supposedly, Player 2's strategy \( y \) can be found similarly by solving the dual problem}. 


\end{document}
