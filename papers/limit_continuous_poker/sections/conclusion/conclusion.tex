\documentclass[../../main/main.tex]{subfiles}
\begin{document}
\section{Conclusion}

We have introduced and analyzed Limit Continuous Poker (LCP), a parametric family of simplified poker games that bridges the gap between Fixed-Bet Continuous Poker (FBCP) and No-Limit Continuous Poker (NLCP). By imposing lower and upper bounds $L$ and $U$ on bet sizes, LCP creates a rich spectrum of strategic environments that interpolate continuously between the fixed-bet and no-limit extremes.

\subsection{Summary of Key Results}

Our analysis has yielded several main contributions:

\textbf{Nash Equilibrium Characterization:} We first described concepts of monotonicity and monotone-admissibility to characterize a notion of optimality, then derived the unique Nash equilibrium of LCP satisfying this condition. In this Nash equilibrium, the bettor partitions hands into bluffing, checking, and value betting regions, with bet sizes varying continuously within the bluffing and value betting ranges. The caller responds with a calling threshold as a function of bet size.

\textbf{Game Value Formula:} We computed the value of LCP as a rational function of the betting limits, obtaining the surprisingly compact expression
\[
V(r,t) = \frac{1 - r^3 - t^3}{14 - 2r^3 - 2t^3}
\]
in the transformed coordinates $r = L/(1+L)$ and $t = 1/(1+U)$. This formula reveals a remarkable symmetry: $V(r,t) = V(t,r)$, meaning that swapping the roles of minimum and maximum bet constraints (in a specific reciprocal sense) leaves the game value unchanged. We proved that the value is monotonically increasing in $U$ and decreasing in $L$, confirming the intuition that more betting flexibility favors the bettor.

\textbf{Convergence to Limiting Cases:} We established that LCP smoothly converges to both FBCP and NLCP in the appropriate limit regimes. As $L \to B$ and $U \to B$, the strategies and value converge to those of FBCP with fixed bet size $B$. As $L \to 0$ and $U \to \infty$, they converge to those of NLCP. These results validate LCP as a genuine generalization of both classical variants.

\textbf{Parameter Sensitivity and Strategic Dynamics:} Our analysis revealed counterintuitive strategic effects: increasing the upper limit $U$ does not uniformly benefit all bettor hands. While the strongest hands gain from the ability to make larger bets, intermediate-strength hands can suffer because the caller adjusts by becoming more conservative across all bet sizes. This illustrates the complex strategic interdependencies in equilibrium play.

\subsection{Strategic Insights and Connections to Real Poker}

The theoretical results for LCP offer several insights relevant to practical poker strategy:

\textbf{Bet Sizing and Stack Depth:} In real poker, effective stack sizes create implicit upper bounds on bet sizes, analogous to our parameter $U$. Our analysis suggests that deeper stacks (higher $U$) create more strategic complexity and favor skilled players who can exploit the additional betting options. The symmetry property $V(L,U) = V(1/U, 1/L)$ suggests a duality between raising minimum bet requirements and constraining maximum bets.

\textbf{Bluffing Frequency and Bet Size:} The equilibrium strategy confirms poker wisdom that bluffing frequency should be calibrated to value betting. It shows that wider bet size limits should incentivize a wider range of bluff to balance possible value bets, as opposed to always bluffing big to scare off the opponent or always bluffing small to minimize risk. This formal description might inform bluffing frequencies and sizes for a variety of stack sizes in heads-up games.

\textbf{Calling Thresholds and Pot Odds:} The calling function $c(s)$ embodies the pot odds principle: the caller must be offered better odds (a larger pot relative to the cost of calling) to justify calling with weaker hands. The equilibrium precisely balances the bettor's bluffing and value betting frequencies against these pot odds.

\subsection{Limitations}

Several simplifications distinguish LCP from real poker:

\begin{itemize}
    \item \textbf{Single Betting Round:} Real poker involves multiple streets of betting with community cards revealed between rounds, creating dynamic information revelation. LCP models only a single decision point.

    \item \textbf{Uniform Hand Distribution:} We assume hand strengths are uniformly distributed on $[0,1]$. Real poker hand distributions are discrete and non-uniform, with specific card combinations determining hand strength.

    \item \textbf{Perfect Correlation:} In our model, hands are perfectly ordered (if $x > y$, the bettor always wins). Real poker has card removal effects and some hands have non-zero equity even when behind.

    \item \textbf{Symmetric Information:} Both players receive one hand each. Real poker often features asymmetric information structures (e.g., one player knows the other folded a certain range on an earlier street).
\end{itemize}

Despite these limitations, the tractability of LCP enables rigorous analysis that would be impossible in more complex settings, providing a foundation for understanding strategic principles.

\subsection{Future Directions}

Several natural extensions of this work could deepen our understanding of bet sizing in poker:

\textbf{Multiple Betting Rounds:} Extending LCP to multiple streets with information revelation between rounds would capture dynamic aspects of poker strategy. One could study how bet size limits in early rounds affect optimal play in later rounds.

\textbf{Asymmetric Limits:} Our model assumes both players face the same ante and pot size. Investigating games where players have different effective stack sizes (asymmetric $U$ values) could model scenarios common in tournament poker.

\textbf{Non-Uniform Hand Distributions:} Relaxing the uniform distribution assumption to model more realistic hand strength distributions could test the robustness of our results.

\textbf{Discrete Approximations:} Real poker involves discrete bet sizing increments (e.g., betting in whole chips or minimum raise increments). Studying discrete approximations to LCP could bridge the gap between our continuous model and practical applications.

\textbf{Computational Tools:} The closed-form solutions for LCP could be used to validate numerical solvers for more complex poker variants. The smooth parameter dependence makes LCP an ideal test bed for computational game theory algorithms.

\textbf{Multi-Player Extensions:} While our analysis focuses on two-player games, extending to three or more players would introduce new strategic considerations such as collusion, side pots, and positional dynamics.

\end{document}
