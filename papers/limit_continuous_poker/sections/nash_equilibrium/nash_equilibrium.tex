\documentclass[../../main/main.tex]{subfiles}
\begin{document}
\section{Nash Equilibrium Strategy Profile}
\label{sec:nash_equilibrium}

Having established the equilibrium structure in Section \ref{sec:solving_lcp} and derived the indifference equations in Section \ref{subsec:constraints}, we now present the complete solution. The system of equations was solved symbolically using Sympy, yielding closed-form expressions for all threshold values and strategic functions. We prove formally in Appendix \ref{app:nash_equilibrium} that this strategy profile constitutes a Nash equilibrium.

\begin{theorem}[LCP Nash Equilibrium]
    \label{thm:nash_equilibrium}
LCP has a unique Nash equilibrium strategy profile in which the caller's strategy is monotone and the bettor's strategy is monotone-admissible (up to measure zero sets of hands for each player). This strategy profile is given by:

\begin{align*}
    x_{0} &= \frac{3 t^{2} \left(t - 1\right)}{r^{3} + t^{3} - 7}\\
    x_{1} &= \frac{- 2 r^{3} + 3 r^{2} + t^{3} - 1}{r^{3} + t^{3} - 7}\\
    x_{2} &= \frac{r^{3} + t^{3} - 1}{r^{3} + t^{3} - 7}\\
    x_{3} &= \frac{r^{3} - 3 r + t^{3} - 4}{r^{3} + t^{3} - 7}\\
    x_{4} &= \frac{r^{3} + 3 r^{2} - 6 r + t^{3} - 4}{r^{3} + t^{3} - 7}\\
    x_{5} &= \frac{r^{3} + t^{3} + 3 t^{2} - 7}{r^{3} + t^{3} - 7}\\
    b_{0} &= \frac{t^{3}}{r^{3} + t^{3} - 7}\\
    b(s) &= \frac{t^{3} \left(s+1\right)^3 - (3s + 1)}{\left(r^{3} + t^{3} - 7\right) \left(s+1\right)^3}\\
    c(s) &= \frac{r^{3} + t^3 -1 + s \left(r^{3} + t^{3} - 7\right)}{\left(s + 1\right) \left(r^{3} + t^{3} - 7\right)}\\
    v(s) &= \frac{r^{3} + t^{3} -1 + \left(r^{3} + t^{3} - 7\right) \left(2 s^{2} + 4 s + 1\right)}{2 \left(r^{3} + t^{3} - 7\right) \left(s^{2} + 2 s + 1\right)}
\end{align*}

where $r = L/(1+L)$ and $t = 1/(1+U)$\footnote{The change of variables to $(r, t)$ significantly simplifies the expressions compared to the original $(L, U)$ formulation. This transformation reveals underlying symmetries and makes many properties more transparent, as we will see in the analysis of game value and parameter effects.}.
\end{theorem}

Refer back to Section \ref{subsec:nash_equilibrium_structure} for an explanation of how these values fit together to actually form the strategy profile. A proof of this theorem can be found in Appendix \ref{app:nash_equilibrium}.

This solution is more interpretable in graphical form. Figure \ref{fig:strategyprofile} shows the strategy profile for various values of $L$ and $U$ ranging from very lenient ($L=0, U=10$) to very restricted ($L=0.5, U=1$). The more lenient bet size limits model something closer to NLCP, while the more restricted bet size limits model something closer to FBCP. Indeed, we see that the strategy profile for $L=0, U=10$ looks qualitatively similar to the strategy profile of NLCP---we will show in Section \ref{sec:strategic_convergence} that the strategy profile approaches the Nash equilibrium of NLCP as $L$ and $U$ approach $0$ and $\infty$, respectively, and that the strategy profile approaches the Nash equilibrium of FBCP as $L$ and $U$ approach some fixed value $s$ from either side.

\begin{figure}[h!]
    \begin{adjustwidth}{-1in}{-1in}
        \centering
        \begin{minipage}{0.6\textwidth}
            \centering
            \includegraphics[width=\textwidth]{images/LCP_profile_0_10.png}
        \end{minipage}
        \hspace{0.05\textwidth}
        \begin{minipage}{0.6\textwidth}
            \centering
            \includegraphics[width=\textwidth]{images/LCP_profile_0.1_2.png}
        \end{minipage}
        \vspace{0.5cm}\\
        \begin{minipage}{0.6\textwidth}
            \centering
            \includegraphics[width=\textwidth]{images/LCP_profile_0.3_1.5.png}
        \end{minipage}
        \hspace{0.05\textwidth}
        \begin{minipage}{0.6\textwidth}
            \centering
            \includegraphics[width=\textwidth]{images/LCP_profile_0.5_1.png}
        \end{minipage}
    \end{adjustwidth}
    \caption{Nash equilibrium strategy profiles for different values of $L$ and $U$, from very lenient to very restricted bet sizes. The bet function maps hand strengths to bet sizes, while the call function gives the minimum calling hand strength for a given bet size. The shaded regions represent the hand strengths for which the caller should call a given bet size.}
    \label{fig:strategyprofile}
\end{figure}


\end{document}
