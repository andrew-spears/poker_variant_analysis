\documentclass[../../main/main.tex]{subfiles}
\begin{document}
\section{Solving for Nash Equilibrium}
\label{sec:solving_lcp}

In this section, we develop the methodology for computing the Nash equilibrium of LCP. Our approach proceeds in three stages: (1) establishing the concept of \textit{monotone calling strategies} and using admissibility to select among multiple equilibria, (2) characterizing the structure that any Nash equilibrium must satisfy, and (3) deriving a system of equations whose solution yields the equilibrium strategy profile. The complete derivation and verification that our solution constitutes a Nash equilibrium is provided in Appendix \ref{app:nash_equilibrium}.

\subsection{Uniqueness and Equilibrium Selection}

Like FBCP, LCP has an infinite class of Nash equilibria, differentiated primarily by how the bettor sizes their bluffs. We resolve this non-uniqueness by imposing two natural refinements: first, we restrict the caller to \textit{monotone} strategies which are more robust and less exploitable; second, we require the bettor's strategy to be admissible (not weakly dominated) against all such monotone calling strategies. These refinements uniquely determine the equilibrium we analyze.

\subsection{Monotone Strategies}
\label{subsec:monotone_strategies}

\begin{definition}[Monotone Calling Strategy]
    A \textit{monotone} calling strategy is a pure strategy which satisfies two conditions:
    \begin{enumerate}
        \item For any bet size $s$ and any two hand strengths $y_1 < y_2$, if the caller calls a bet of size $s$ with $y_1$, they must also call with $y_2$.
        \item For any hand strength $y$ and any two bet sizes $s_1 < s_2$, if the caller calls a bet of size $s_2$ with $y$, they must also call a bet of size $s_1$ with $y$.
    \end{enumerate}
\end{definition}

This should sound intuitive. Clearly, calling with a stronger hand is weakly better than calling with a weaker hand. Restricting to pure strategies can be explained similarly - it is better to always call with a stronger hand and always fold a weaker one than to mix between the two.

Violating the first condition (in a non-negligible way) is actually weakly dominated - not only is a monotone strategy weakly better against all opponents, but there exists an opponent against which the non-monotone strategy is strictly better (see Appendix \ref{sec:monotone_proofs} for proof).

The second condition for a monotone calling strategy - that the caller must be more willing to call smaller bets - is more subtle. From a poker player's perspective, it aligns with intuition about pot odds: a larger bet is riskier and should require a stronger hand to call. While violating this condition is not dominated, it leads to exploitable calling strategies. If the caller calls less aggressively against smaller bets, the bettor can take smaller risks for higher returns. A monotone calling strategy is therefore less exploitable, and imposing this condition yields a unique Nash equilibrium.

\begin{definition}[Monotone-Admissible Strategy]
    A betting strategy $\sigma_B$ is \textit{monotone-admissible} if it is admissible in LCP against the set of monotone calling strategies. That is, there does not exist a betting strategy $\sigma_B'$ that performs at least as well against all monotone calling strategies and strictly better against at least one.
\end{definition}

This definition distinguishes bettor strategies that differ only in how they bluff. The hand strength of a bluff is irrelevant when the caller plays optimally, since the caller never calls with a hand weaker than any bluff. However, if the caller deviates to a suboptimal but still monotone strategy, the bettor's bluffing hand strength matters. Monotone-admissibility selects the equilibrium where the bettor bluffs larger with weaker hands and smaller with stronger hands, which is optimal against all monotone deviations. See Appendix \ref{sec:monotone_proofs} for detailed analysis.

\subsection{Nash Equilibrium Structure}
\label{subsec:nash_equilibrium_structure}

We will now describe the structure of the Nash equilibrium in terms of constants $x_i$ and functions $c(s)$, $b(s)$, and $v(s)$. These turn out to be fully determined by the parameters $L$ and $U$, but for now they are unknown. Notice that both players use pure strategies, like in NLCP: the bettor maps hand strengths directly to bet sizes, and the caller maps hand strengths and bet sizes to actions with no mixing.

    \begin{enumerate}
        \item The caller has a calling threshold $c(s)$ that is continuous in $s$, including at endpoints $L$ and $U$. They call with hands $y \geq c(s)$ and fold with hands $y < c(s)$\footnote{The action taken at the threshold is irrelevant, since it occurs with probability zero.}.
        \item The bettor partitions $[0,1]$ into three regions: bluffing $x \in [0,x_2]$, checking $x \in [x_2,x_3]$, and value betting $x \in [x_3,1]$.
        \item Within the bluffing region, the bettor's partitions into a max-betting region $x \in [x_0,x_1]$, an intermediate region $x \in [x_1,x_2]$, and a min-betting region $x \in [x_2,x_3]$.
        \item Within the intermediate bluffing region, the bettor bets according to a continuous, decreasing function $s=b^{-1}(x)$ with endpoints $b^{-1}(x_0)=U$ and $b^{-1}(x_3)=L$.
        \item Within the value betting region, the bettor partitions into a min-betting region $x \in [x_3,x_4]$, an intermediate region $x \in [x_4,x_5]$, and a max-betting region $x \in [x_5,1]$.
        \item Within the intermediate value betting region, the bettor bets according to a continuous, increasing function $s=v^{-1}(x)$ with endpoints $v^{-1}(x_3)=L$ and $v^{-1}(x_5)=U$.
    \end{enumerate}



% \begin{customproof}
%     We will prove the structure of the Nash equilibrium by establishing each claim:
%     % \begin{enumerate}
%     %     \item The caller must use a threshold strategy with cutoff $c(s)$
%     %     \item The cutoff $c(s)$ must be continuous, even at endpoints $L$ and $U$
%     %     \item The bettor must bet for value with hands stronger than $c(s)$, bluff with hands weaker than $c(s)$, and check some range of intermediate hands
%     %     \item Value betting sizes must increase with hand strength
%     %     \item When bluffing, the bettor must be indifferent among exactly the sizes which are used for value betting
%     %     \item Bluffing sizes should decrease with hand strength
%     % \end{enumerate}

%     \textbf{Claim 1:} The caller must use a threshold strategy with cutoff $c(s)$. This follows from lemma \ref{lem:monotone_dominated}.

%     % \textbf{Claim 2:} The cutoff $c(s)$ must be non-decreasing in $s$.
%     % If $c(s)$ were decreasing at any point, the bettor could exploit this by bluffing with a slightly smaller size than they would otherwise use. This would cause the caller to fold more often while risking less money, contradicting equilibrium.

%     \textbf{Claim 2:} The cutoff $c(s)$ must be continuous, even at endpoints $L$ and $U$.
%     If $c(s)$ had a discontinuity, the bettor's expected value from bluffing would also be discontinuous. They could then exploit by bluffing with sizes just below the discontinuity, contradicting equilibrium.

%     \textbf{Claim 3:} Value bets must be made with hands stronger than $c(s)$, bluffs with hands weaker than $c(s)$.
%     By definition, when value betting size $s$, the bettor wants to get called by weaker hands (requiring $x > c(s)$). When bluffing size $s$, the bettor wants to get called by stronger hands (requiring $x < c(s)$).

%     \textbf{Claim 4:} Value betting sizes must increase with hand strength.
%     Since $c(s)$ is non-decreasing, stronger hands can profitably bet larger sizes and get called by a more restricted range of strong hands. Weaker value betting hands must bet smaller to get called by a wider range.

%     \textbf{Claim 5:} The bettor must be indifferent among bluffing sizes that are also used for value bets.
%     The expected value of bluffing size $s$ is:
%     \begin{equation}
%         \mathbb{E}[\text{bluff } s] = c(s) - (1-c(s)) \cdot s
%     \end{equation}
%     This is independent of the bettor's hand strength. If the bettor strictly preferred certain sizes for bluffing, they would never bluff with other sizes. If any other size were used for value betting, then the caller could exploit by only calling those sizes with hands stronger than the value betting range.
%     Additionally, the bettor cannot bluff with sizes which are not used for value betting. In this case, the caller can similarly exploit by always calling this size with hands stronger than the bluffing range.

%     \textbf{Claim 6:} Bluffing sizes should decrease with hand strength. This claim is not necessary to have a Nash equilibrium, but it is what makes the bettor's strategy monotone-admissible.
%     If the caller deviates by calling too loosely but maintains consistency (never calling with weaker hands and folding with stronger hands to the same bet size), the bettor uniquely benefits by bluffing larger with their weakest hands and bluffing smaller with their strongest hands. This gives them a possibility of winning showdowns with their strongest bluffing hands, which would not happen if they bluffed large with the strongest hands.
%     \todo{admissiblity?}
% \end{customproof}

See figure \ref{fig:strategyprofile} for visual representations of the strategy profile.

\subsection{Constraints and Indifference Equations}
\label{subsec:constraints}

Having established the qualitative structure of the Nash equilibrium, we now derive the quantitative relationships that the strategy profile must satisfy. These constraints arise from two fundamental equilibrium conditions: players must be indifferent among actions they mix over (or, in our case, use with positive probability), and players must optimize when choosing among available actions. The resulting system of differential and algebraic equations will uniquely determine all threshold values $x_0, \ldots, x_5$ and the functions $b(s)$, $v(s)$, and $c(s)$.

The Nash equilibrium strategy profile must satisfy several constraints and indifference conditions, which we will derive and use to solve for the strategy profile. The key conditions are:

\begin{itemize}
    \item The caller must be indifferent between calling and folding at their calling threshold
    \item The bettor must be indifferent between checking and betting at their value betting and bluffing thresholds
    \item The bettor's bet size for a value bet must maximize their expected value
    \item The bettor's strategy must be continuous in bet size (in the regions where they bet)
\end{itemize}

These conditions give us the following system of equations: \label{eq:nash_equilibrium_system}

\textbf{Caller Indifference:}
\begin{align}
    & (x_2-x_1) \cdot (1+L) - (x_4-x_3) \cdot L = 0 \\
    & x_0 \cdot (1+U) - (1-x_5) \cdot U = 0\\
    & |b'(s)| \cdot (1 + s) - |v'(s)| \cdot s = 0
\end{align}

\textbf{Bettor Indifference and Optimality:}
\begin{align}
    & -sc'(s) - c(s) + 2 v(s) - 1 = 0 \label{eq:valueoptimality}\\
    & (x_3-c(L)) \cdot (1+L) - (1-x_3) \cdot (L) + c(L) = x_3 \label{eq:valueindiff}\\ 
    & c(s) - (1-c(s)) \cdot s = x_2 \label{eq:bluffindiff}
\end{align}

\textbf{Continuity Constraints:}
\begin{align}
    & b(U) = x_0 \\
    & b(L) = x_1 \\
    & v(U) = x_5 \\
    & v(L) = x_4.
\end{align}

We will now derive each of these equations in turn. Note that for this analysis, it is simpler to pretend that payoffs exclude the initial ante of 0.5, since this is a sunk cost to both players and we only care about relative payoffs between actions.

\subsubsection{Caller Indifference}
\label{subsec:caller_indifference}

By definition, $c(s)$ is the threshold above which the caller calls and below which they fold. This means that in Nash Equilibrium, the caller must be indifferent between calling and folding with a hand strength of $c(s)$:


  \[  \mathbb{E}[\text{call } c(s)] = \mathbb{E}[\text{fold } c(s)] \]
  \[  \mathbb{P}[\text{bluff} | s] \cdot (1+s) - \mathbb{P}[\text{value bet} | s]\cdot s = 0. \]

We now split into cases based on the value of $s$.

\textbf{Case 1: $s = L$}. The hands the bettor value bets $L$ with are $x \in (x_3, x_4)$, and the hands they bluff with are $x \in (x_1, x_2)$. 

\begin{equation}{\label{callindiffmin}}
    (x_2-x_1) \cdot (1+L) - (x_4-x_3) \cdot L = 0.
\end{equation}

Here, we are implicitly multiplying both sides by the common denominator of $(x_4-x_3) + (x_2-x_1)$.

\textbf{Case 2: $s = U$}. The hands the bettor value bets $U$ with are $x \in (x_5, 1)$, and the hands they bluff with are $x \in (0, x_0)$. 

\begin{equation}{\label{callindiffmax}}
    (1-x_5) \cdot (1+U) - x_0 \cdot U = 0,
\end{equation}

again, implicitly multiplying both sides by the common denominator of $(1-x_5) + x_0$.

\textbf{Case 3: $L \leq s \leq U$}. In this case, the bettor has exactly one value hand and one bluffing hand, but somewhat paradoxically, they are not equally likely. The probability of a value bet given the size $s$ is related to the inverse derivative of the value function $v(s)$ at $s$, and the same goes for a bluff. This gives us the following relation:

\[ \frac{\mathbb{P}[\text{value bet} | s]}{\mathbb{P}[\text{bluff} | s]} = \frac{|b'(s)|}{|v'(s)|}\]

An intuitive interpretation of this is that for any small neighborhood around the bet size $s$, the bettor has more hands which use a bet size in the neighborhood if $v(s)$ does not change rapidly around $s$, that is, if $|v'(s)|$ is small. The same goes for bluffing hands, and as we limit the neighborhood to a single point, the ratio of the two probabilities approaches the ratio of the derivatives. We know that these are the only two possible bettor actions for such a bet size, so

\[ \mathbb{P}[\text{value bet} | s] = \frac{|b'(s)|}{|b'(s)| + |v'(s)|} \]
\[ \mathbb{P}[\text{bluff} | s] = \frac{|v'(s)|}{|b'(s)| + |v'(s)|} \]

Plugging this into the indifference equation and dividing out the common denominator, we get:

\begin{equation}{\label{callindiff}}
    |b'(s)| \cdot (1 + s) - |v'(s)| \cdot s = 0.
\end{equation}



\subsubsection{Bettor Indifference and Optimality}

When the bettor makes a value bet, they are attempting to maximize the expected value of the bet. We can write the expected value of a value bet as:

\begin{align*}
    \mathbb{E}[\text{value bet } s | x] & = \mathbb{P}[\text{call w ith worse}] \cdot (1+s) - \mathbb{P}[\text{call with better}] \cdot s + \mathbb{P}[\text{fold}] \cdot 1 \\
    & = (x-c(s)) \cdot (1+s) - (1-x) \cdot (s) + c(s).
\end{align*}

To maximize this, we take the derivative with respect to $s$ and set it equal to zero. Crucially, we are treating $c(s)$ as a function of $s$ and using the chain rule, since changing the bet size $s$ will also change the calling threshold $c(s)$. We want this optimality condition to hold for the bettor's Nash equilibrium strategy, so we set $x=v(s)$. This gives us:

\begin{align}{\label{valueoptimality}}
    \nonumber \frac{d}{ds} \mathbb{E}[\text{value bet } s | x=v(s)] & = 0 \\
    -sc'(s) - c(s) + 2 v(s) - 1 & = 0.
\end{align}

Additionally, when the bettor has the most marginal value betting hand at $x=x_3$, they should be indifferent between a minimum value bet and a check: 

\begin{align}{\label{valueindiff}}
    \nonumber \mathbb{E}[\text{value bet } L | x=x_3] & = \mathbb{E}[\text{check} | x=x_3]\\ 
    (x_3-c(L)) \cdot (1+L) - (1-x_3) \cdot (L) + c(L) & = x_3.
\end{align}

Finally, when the bettor has the most marginal bluffing hand at $x=x_2$, they should be indifferent between a minimum bluff and a check. However, as we discussed earlier, the bettor should be indifferent among all bluffing sizes, so the bettor should actually be indifferent between checking and making any bluffing size $s$ at $x=x_2$. This gives us:

\begin{align}{\label{bluffindiff}}
    \nonumber \mathbb{E}[\text{bluff } s | x=x_2] & = \mathbb{E}[\text{check} | x=x_2]\\ 
    c(s) - (1-c(s)) \cdot s & = x_2.
\end{align}

\subsubsection{Continuity Constraints}

As discussed above, the bettor's strategy is continuous in $s$ and $x$ (except when checking). This means that the endpoints of the functions $v(s)$ and $b(s)$ are constrained as follows:

\begin{equation}{\label{continuityconstraints}}
	 b(U) = x_0, \;\; b(L) = x_1, \;\; v(U) = x_5, \;\; v(L) = x_4.
\end{equation}

\end{document}