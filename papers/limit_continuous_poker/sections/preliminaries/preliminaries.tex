\documentclass[../../main/main.tex]{subfiles}
\begin{document}
\section{Preliminaries and Conventions}

\subsection{Game-Theoretic Concepts}

All continuous poker variants studied in this paper are \textit{two-player zero-sum games}. LCP also has infinite action and type spaces (continuous bet sizes and hand strengths). Unlike finite games, in an infinite game, a Nash equilibrium may or may not exist. For LCP, we explicitly construct a Nash equilibrium in Section \ref{sec:nash_equilibrium} and verify in Appendix \ref{app:nash_equilibrium} that no player can improve by unilateral deviation. This constructive approach establishes existence without appealing to general existence theorems.

Once a Nash equilibrium is established, the game has a well-defined \textit{value}: the expected payoff to the bettor at equilibrium. In two-player zero-sum games, all Nash equilibria yield the same payoff to each player, so this value is unique. Since the game is zero-sum, the caller's expected payoff is the negative of this value. A positive value indicates an advantage for the bettor. We find the value of LCP in Section \ref{sec:game_value} by analyzing payoffs in the constructed Nash equilibrium.

As is standard in continuous games, we define uniqueness of strategies up to sets of measure zero. For instance, if two strategies take identical actions except at an exact threshold hand strength (which occurs with probability 0), then they are equivalent for our purposes.

\subsection{Conventions}

\begin{description}[style=nextline, leftmargin=1em, font=\bfseries]
    \item[Ante and Pot Size:] Each player contributes an ante of 0.5 units, creating an initial pot of 1 unit. All bet sizes are measured in units relative to this pot. This convention (ante = 0.5 rather than ante = 1) simplifies payoff calculations.
    
    \item[Payoffs:] Payoffs represent the net gain or loss relative to the initial ante. A check results in the winner receiving the pot of 1 minus their ante of 0.5, for a net payoff of $\pm 0.5$. When a bet of size $s$ is called, the winner receives $1 + 2s$ (pot plus both contributions) minus their ante of 0.5, for a net payoff of $\pm(0.5 + s)$.
    
    \item[Inequalities and Measure Zero:] When we write, for example, ``the caller calls with hands $y \geq c(s)$,'' the choice of $\geq$ versus $>$ is immaterial because the set $\{y : y = c(s)\}$ has measure zero. We use closed intervals $[a,b]$ and open intervals $(a,b)$ interchangeably when the boundary points have probability zero.
    
    \item[Monotone Strategies:] A calling strategy is \textit{monotone} if (1) stronger hands are more likely to call for any fixed bet size, and (2) smaller bets are more likely to be called for any fixed hand strength. This concept is formalized in Section 3.
    
    \item[Admissibility:] A strategy is \textit{admissible} if it is not weakly dominated by any other strategy. Among multiple Nash equilibria, we focus on those where the bettor's strategy is admissible against all monotone calling strategies (Section 3).
\end{description}

\end{document}
