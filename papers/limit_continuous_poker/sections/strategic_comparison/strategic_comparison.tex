\documentclass[../../main/main.tex]{subfiles}
\begin{document}
\section{Strategic Comparison to Fixed-Bet and No-Limit Continuous Poker}
\label{sec:strategic_comparison}

LCP interpolates between Fixed-Bet Continuous Poker (FBCP) and No-Limit Continuous Poker (NLCP). We now make this precise by showing that the LCP strategies converge to those of FBCP and NLCP as the betting limits approach appropriate boundary values.

\subsection{Notation}

We use the following notation for strategy functions across variants:

\begin{itemize}
    \item $S_{FB}(x, B)$, $C_{FB}(s, B)$: Bettor and caller strategies in FBCP with fixed bet $B$
    \item $S_{NL}(x)$, $C_{NL}(s)$: Bettor and caller strategies in NLCP
    \item $S_{LCP}(x, L, U)$, $C_{LCP}(s, L, U)$: Bettor and caller strategies in LCP
\end{itemize}

The FBCP strategies are:
\begin{align*}
	S_{FB}(x, B) & = \begin{cases}
    B & x < \frac{B}{(1+2B)(2+B)} \text{ (bluff)}\\
    0 & \frac{B}{(1+2B)(2+B)} < x < \frac{1 + 4B + 2B^2}{(1+2B)(2+B)} \text{ (check)}\\
    B & x > \frac{1 + 4B + 2B^2}{(1+2B)(2+B)} \text{ (value)}
\end{cases}, \quad
C_{FB}(B) = \frac{B(3 +2B)}{(1+2B)(2+B)}.
\end{align*}

The NLCP strategies are:
\begin{align*}
    S_{NL}(x) &= \begin{cases}
        b_{NL}^{-1}(x) & x < \frac{1}{7} \text{ (bluff)} \\
        0 & \frac{1}{7} < x < \frac{4}{7} \text{ (check)} \\
        v_{NL}^{-1}(x) & x > \frac{4}{7} \text{ (value)}
    \end{cases}, \quad
    C_{NL}(s) = 1 - \frac{6}{7 (s+1)},
\end{align*}
where $v_{NL}(s) = 1 - \frac{3}{7(s+1)^2}$ and $b_{NL}(s) = \frac{3s+1}{7(s+1)^3}$.

\subsection{Strategic Convergence}
\label{sec:strategic_convergence}

The LCP strategy functions are rational in $L$ and $U$, so convergence follows from continuity by direct substitution of limit values.

\begin{remark}[Convergence to FBCP]
As $L, U \to B$ for any fixed $B > 0$:
\[
S_{LCP}(x, L, U) \to S_{FB}(x, B), \quad C_{LCP}(s, L, U) \to C_{FB}(s, B).
\]
At $L = U = B$, the intermediate bet-size regions collapse ($x_0 = x_1$ and $x_4 = x_5$), leaving only the fixed bet $B$ as an option. The thresholds reduce to:
\[
x_2|_{B,B} = \frac{B}{(1+2B)(2+B)}, \quad x_3|_{B,B} = \frac{2B^2+4B+1}{(1+2B)(2+B)},
\]
which match the FBCP bluff/check and check/value boundaries. The calling threshold $c(s)|_{B,B} = \frac{B(3+2B)}{(1+2B)(2+B)} = C_{FB}(B)$.
\end{remark}

\begin{remark}[Convergence to NLCP]
As $L \to 0$ and $U \to \infty$:
\[
S_{LCP}(x, L, U) \to S_{NL}(x), \quad C_{LCP}(s, L, U) \to C_{NL}(s).
\]
At these limits, the boundary regions collapse ($x_0 \to 0$, $x_5 \to 1$), and the thresholds become:
\[
x_1 = x_2 \to \frac{1}{7}, \quad x_3 = x_4 \to \frac{4}{7},
\]
matching the NLCP bluff/check and check/value boundaries. The bet-size functions converge to
\[
b(s) \to \frac{3s+1}{7(s+1)^3}, \quad v(s) \to 1 - \frac{3}{7(s+1)^2},
\]
which are exactly $b_{NL}(s)$ and $v_{NL}(s)$. The calling threshold converges to $c(s) \to 1 - \frac{6}{7(1+s)} = C_{NL}(s)$.
\end{remark}

These convergence results confirm that LCP provides a smooth interpolation between the two benchmark models.

\end{document}
