\documentclass[../../main/main.tex]{subfiles}
\usepackage{geometry}

\begin{document}
\section{Parameter and Payoff Analysis}
\label{sec:payoff_analysis}

Having established the Nash equilibrium (Section \ref{app:nash_equilibrium}), analyzed the game value (Section \ref{sec:game_value}), and proved convergence to FBCP and NLCP (Section \ref{sec:strategic_comparison}), we now explore in greater detail how the parameters $L$ and $U$ affect player strategies and payoffs. This section summarizes key insights and presents visualizations that illuminate the strategic dynamics of LCP. Complete technical proofs are provided in Appendices \ref{sec:parameter_analysis} and \ref{sec:payoff_analysis_proofs}.

\subsection{Visualizing Payoffs in Equilibrium}

In Nash equilibrium, each hand combination $(x, y)$ uniquely determines the bettor's payoff. Figure \ref{fig:payoffs} shows how these payoffs vary across the unit square for different values of $L$ and $U$, from strict limits ($L=U=1$) to lenient limits ($L=0, U \to \infty$).

\clearpage
\newgeometry{top=0in,left=0.5in,right=0.5in,bottom=0.5in}
\begin{figure}[p]
    \begin{adjustwidth}{-1.25in}{-1.25in}
        \centering
        \begin{minipage}{0.4\textwidth}
            \centering
            \includegraphics[width=\textwidth]{../payoff_analysis/images/payoffs_1_1.png}
        \end{minipage}
        \hspace{0.05\textwidth}
        \begin{minipage}{0.4\textwidth}
            \centering
            \includegraphics[width=\textwidth]{../payoff_analysis/images/payoffs_0.9_1.1.png}
        \end{minipage}
        \vspace{0.4cm}\\
        \begin{minipage}{0.4\textwidth}
            \centering
            \includegraphics[width=\textwidth]{../payoff_analysis/images/payoffs_0.5_1.5.png}
        \end{minipage}
        \hspace{0.05\textwidth}
        \begin{minipage}{0.4\textwidth}
            \centering
            \includegraphics[width=\textwidth]{../payoff_analysis/images/payoffs_0.3_2.png}
        \end{minipage}
        \vspace{0.4cm}\\
        \begin{minipage}{0.4\textwidth}
            \centering
            \includegraphics[width=\textwidth]{../payoff_analysis/images/payoffs_0.1_5.png}
        \end{minipage}
        \hspace{0.05\textwidth}
        \begin{minipage}{0.4\textwidth}
            \centering
            \includegraphics[width=\textwidth]{../payoff_analysis/images/payoffs_0_inf.png}
        \end{minipage}
    \end{adjustwidth}
    \caption{Bettor payoffs in Nash equilibrium as a function of hand strengths $x, y$ for fixed bet size $B=1$ (top left), and No-Limit Continuous Poker (bottom right). Intermediate plots show the payoffs for different values of $L$ and $U$ between the two extremes. Regions are labeled according to the outcome of the game in Nash equilibrium.}
    \label{fig:payoffs}
\end{figure}

\restoregeometry

The visualization reveals that the biggest wins and losses occur when both hands are strong (top right), consistent with real poker intuition. Large payoffs also occur when a very weak bettor bluffs big and gets called by a strong caller (top left). As limits become more lenient, these extreme outcomes become more pronounced but also less likely, since making and calling maximum bets become riskier for both players.

It is worth noting that the bettor's strongest hands (right edge) actually become less likely to make any profit (more than the ante) as limits increase. These strongest hands make very large bets, which force all but the strongest hands to fold, but win huge pots when they do get called.

In more complicated poker variants, it is common to ``slowplay" strong hands by checking or making small bets to induce bluffs from the opponent. In LCP, there is only one betting round and the caller is not allowed to raise, both of which make slowplaying obsolete. With extremely strong hands, the benefit of winning a large pot when betting big outweighs the lower likelihood of getting called.

This strategic pattern demonstrates a fundamental tension in poker: extracting maximum value from strong hands requires finding the optimal balance between bet size (which determines pot size when called) and calling frequency (which decreases as bet size increases). In LCP, the strongest hands resolve this tension by accepting a lower calling frequency in exchange for winning much larger pots.


\subsection{Expected Value by Hand Strength}

Beyond specific hand matchups, we can analyze the expected value $EV(x)$ of a bettor hand $x$ averaged over all possible caller hands. This function characterizes how profitable each hand is in equilibrium (intuitively, how happy the bettor should be to see each specific hand).

We can calculate $EV(x)$ analytically by combining previous results:

\begin{equation}
    \label{ev_x}
    EV(x) = \begin{cases}
        x_2-\frac{1}{2} & \text{if } x \leq x_2 \\
        x-\frac{1}{2} & \text{if } x_2 < x \le x_3 \\
        x(2L + 1) - L(c(L) + 1) - \frac{1}{2} & \text{if } x_3 < x < v(L) \\
        x(2v^{-1}(x) + 1) - v^{-1}(x)(c(v^{-1}(x)) + 1) - \frac{1}{2} & \text{if } v(L) \leq x \leq v(U) \\
        x(2U + 1) - U(c(U) + 1) - \frac{1}{2} & \text{if } x > v(U).
    \end{cases}
\end{equation}

\begin{figure}[h!]
    \centering
    \includegraphics[width=\textwidth]{../payoff_analysis/images/ExpectedPayoffs.png}
    \caption{Expected value of bettor hand strength $x$ under optimal play in LCP.}
    \label{fig:ev_x}
\end{figure}

This function is much easier to view graphically (Figure \ref{fig:ev_x}). Observations:
\begin{itemize}
    \item The function $EV(x)$ is increasing in $x$ (Appendix \ref{sec:payoff_analysis}, Theorem \ref{thm:ev_increasing}).
    \item All bluffing hands ($x \leq x_2$) achieve the same expected value $x_2 - 1/2$, regardless of hand strength. This is because the caller will never call with a losing hand, so a bluff either induces a fold or loses the hand. If a bluffing action had lower EV, the bettor would simply not take this action, so all bluffing hands must have equal EV in equilibrium.
    \item Checking hands ($x_2 < x \leq x_3$) have EV equal to $x-1/2$. Notice the slope of 1  from the pot being exactly 1 unit.
    \item Value betting hands ($x > x_3$) earn more steeply increasing returns as hand strength increases. This is because the pot is increasing in size as the bettor makes larger bets with stronger hands, so a marginal increase in hand strength becomes increasingly more profitable.
\end{itemize}

\subsection{Effect of Increasing the Upper Limit $U$}

A counterintuitive result emerges when examining how individual hand values change as we increase $U$: for strong hands, the expected value \emph{decreases} beyond a certain threshold of $U$ (see Figure \ref{fig:ev_x_vs_U}). This occurs despite the bettor having strictly more strategic options.

Notice that this result is not contradictory to the previous section. There are three distinct statements being made:

\begin{itemize}
    \item For any fixed parameters $L, U$, stronger hands have higher expected value (Figure \ref{fig:ev_x} and Theorem \ref{thm:ev_increasing}). This must be true because the bettor could always play as though they had the weaker hand.
    \item Averaged across all hand strengths $x \in [0, 1]$, increasing $U$ or decreasing $L$ increases the bettors' overall expected value. This must be true because the bettor could artificially restrict their bet sizes to a narrower range.
    \item For a fixed hand strength $x$, increasing $U$ can \emph{decrease} the expected value of that specific hand (Figure \ref{fig:ev_x_vs_U}). This is the counterintuitive result being discussed here.
\end{itemize}

\begin{figure}[h!]
    \centering
    \includegraphics[width=\textwidth]{../sections/payoff_analysis/images/ev_vs_U.png}
    \caption{Expected value of a value-betting hand $x$ versus the upper limit $U$ under optimal play. Each curve increases in $U$ up to some peak, after which it decreases, indicating that more flexibility decreases the expected value of that specific hand.}
    \label{fig:ev_x_vs_U}
\end{figure}

The explanation lies in strategic interdependence: as $U$ increases, the bettor can make larger bets with their strongest hands, which forces the caller to become more conservative across \emph{all} bet sizes. This defensive adjustment by the caller harms the expected value of intermediate-strength hands, even though they're betting less. Only the very strongest hands (above a threshold greater than $v(U)$) benefit from the increased flexibility, and the threshold for `strongest' only increases further as we increase $U$.

We can state this formally:

\begin{theorem}
    \label{thm:payoff_increasing}
    For any any $0 < L \leq U$ and any value-betting hand strength $x > x_3$, the derivative $\frac{d}{dU} EV(x) < 0$ if

    $$x < \max\left(v(U), \frac{1}{2(1+U)} \left( U \frac{\partial x_2}{\partial U} + \frac{U^2 + x_2(1 + 2U)}{1+U} \right) \right),$$

    and $\frac{d}{dU} EV(x) > 0$ otherwise.
\end{theorem}

This is equivalent to the statement above. The expression on the right is the threshold on $x$ below which the expected value is decreasing in $U$. This results relies on several others, which are stated below:

\begin{lemma}[Monotonicity in $U$]\label{thm:monotonicity-U}
    For fixed $L$ and $x \in [x_3, v(U)]$, as $U$ increases:
    \begin{enumerate}
        \item The bluffing threshold $x_2$ increases (more hands bluff).
        \item The bet size $v^{-1}(x)$ decreases.
        \item The calling cutoff for the chosen bet size $c(v^{-1}(x))$ increases despite smaller bets.
    \end{enumerate}
\end{lemma}

A complete analysis and proof can be found in Appendix \ref{sec:parameter_analysis}. The crucial takeaway is that varying bet size limits has complex effects on optimal strategies and payoffs, and these effects are not always intuitive.

\end{document}
