\documentclass[../../main/main.tex]{subfiles}
\begin{document}
\section{Limit Continuous Poker (LCP)}
\label{sec:lcp}

Limit Continuous Poker (LCP) is an extension of NLCP with the additional constraint of limiting bet sizes to a range. Just like NLCP, both players are dealt random hand strengths between 0 and 1. The bettor acts first, choosing either to check or to make a bet from an allowed range of sizes. If a bet is made, the caller must decide whether to call the bet or fold. 

\begin{definition}[LCP]
A two-player zero-sum game where:
\begin{itemize}
    \item The bettor and caller are each dealt independent hand strengths $X, Y \sim \text{Uniform}[0,1]$
    \item The bettor observes their hand strength $x$ and chooses an action from $\mathcal{A}_1 = \{\text{0}\} \cup [L, U]$ (a bet of 0 is a check)
    \item If the bettor chooses an action from $[L, U]$ (a bet), then the caller observes the bettor's action along with their own hand strength $y$ and chooses from $\mathcal{A}_2 = \{\text{call}, \text{fold}\}$ 
    \item Payoffs are determined as follows:
    \begin{itemize}
        \item If the bettor checks: payoff is $0.5$ to the player with higher hand strength (the pot of 1, minus the initial ante of 0.5)
        \item If the bettor bets $s \in [L, U]$ and the caller calls: payoff is $0.5 + s$ (the pot of $1+2s$ minus the initial ante of 0.5 and the bettor's contribution of $s$) to the player with higher hand strength
        \item If the bettor bets $s \in [L, U]$ and the caller folds: payoff is $0.5$ to the bettor
    \end{itemize}
\end{itemize}

We can describe a strategy for the bettor as a measurable function $\sigma_1: [0,1] \to \mathcal{A}_1$ mapping hand strengths to actions. A strategy for the caller is a measurable function $\sigma_2: [L, U] \times [0, 1] \to \mathcal{A}_2$ mapping bet sizes and hand strengths to caller responses.
\end{definition}

The motivation for studying this variant is twofold. First, it is a more realistic variant than NLCP or FBCP, where bets are not fixed but are also not unbounded. In most real variants of poker, bet sizes are constrained by the stack sizes of the players and by a minimum bet size. Strong poker players have intuition about how bet size constraints affect strategy, but rigorously proving this intuition is often impossible given the combinatorial complexity of the game. 

Second, LCP can be seen as a generalization of FBCP and NLCP; specifically, as $L \to 0$ and $U \to \infty$, LCP approaches NLCP, and as $L \to B$ and $U \to B$ for some fixed value $B$, LCP approaches FBCP. Studying LCP can help us understand the relationship between these two and answer questions about why they produce the strategies they do (see Section \ref{sec:strategic_convergence} for formal convergence results).

In the next section, we develop the methodology for solving LCP and describing optimal strategies. Section \ref{subsec:nash_equilibrium_structure} will describe the structure of this equilibrium, and the complete closed-form solution is presented in Section \ref{sec:nash_equilibrium}.

\end{document}