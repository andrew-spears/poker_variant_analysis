\documentclass[../../main/main.tex]{subfiles}
\begin{document}

\section{Payoff Analysis: Complete Proofs}
\label{sec:payoff_analysis}

This appendix provides detailed proofs regarding the expected payoffs and values of different hand strengths in the Nash equilibrium of Limit Continuous Poker.

\subsection{Expected Value of Bettor's Hand Strengths}

In addition to considering payoffs for a specific bettor-caller hand combination, we can also consider the expected value of a given hand for the bettor, not knowing the hand of the caller.

\begin{theorem}
    \label{thm:ev_bettor}
    Let $EV(x)$ denote the expected value of a value-betting hand $x$ in the unique admissible Nash equilibrium:
    \begin{equation}
        EV(x) = \begin{cases}
            x_2-\frac{1}{2} & \text{if } x \leq x_2 \\
            x-\frac{1}{2} & \text{if } x_2 < x \le x_3 \\
            x(2L + 1) - L(c(L) + 1) - \frac{1}{2} & \text{if } x_3 < x < v(L) \\
            x(2v^{-1}(x) + 1) - v^{-1}(x)(c(v^{-1}(x)) + 1) - \frac{1}{2} & \text{if } v(L) \leq x \leq v(U) \\
            x(2U + 1) - U(c(U) + 1) - \frac{1}{2} & \text{if } x > v(U).
        \end{cases}
    \end{equation}
\end{theorem}

\begin{customproof}
    For bluffing hands $x \leq x_2$, we can use a simple argument to show that the hand strength $x$ is actually irrelevant. The bettor never gets called by worse hands, so either the caller folds or calls with the best hand. In either case, the bettor's payoff has no dependence on $x$, so the payoff must be the same for all $x \leq x_2$ (otherwise, the lower-payoff hands would imitate the strategy of the higher-payoff hands). We know that at $x=x_2$, the bettor is indifferent between bluffing and checking, so the payoff must be $x_2-\frac{1}{2}$ for all $x \leq x_2$.

    For checking hands $x_2 \leq x \leq x_3$, the bettor wins only the ante exactly when they have the best hand, which happens with probability $x$. The value of the ante is $\frac{1}{2}$, so the bettor's expected value is $x-\frac{1}{2}$.

    For any value betting hand, the bettor has three cases to consider: the caller folds, the caller calls with a worse hand, or the caller calls with the best hand. We simply sum the expected value of each of these cases.

    \begin{align*}
        EV(x) & = \frac{1}{2} c(s) + (x - c(s)) \left(s+\frac{1}{2}\right) + (1-x) \left(-s-\frac{1}{2}\right)
    \end{align*}

    The last three cases come from substituting $L, v^{-1}(x), U$ for $s$ in the expression above and simplifying.
\end{customproof}

\subsection{Monotonicity of Expected Value}

We can quickly verify that the bettor's expected value is increasing in $x$. This must be the case, since with any given hand strength, the bettor can always choose to imitate the Nash equilibrium strategy of a weaker hand, so the stronger hand must be at least as good in expectation.

\begin{theorem}
    \label{thm:ev_increasing}
    For any fixed $L, U$, the bettor's expected value $EV(x)$ is increasing in $x$.
\end{theorem}

\begin{customproof}
    It is clear from inspection that any checking EV is higher than that of a bluff and that the checking EV is increasing in $x$. We also know that at $x=x_3$, the bettor is indifferent between checking and betting, so the EV of checking and betting must be equal at this point. Therefore, we only need to show that within the checking and value betting regions, the EV is increasing in $x$. This is obvious for checking hands. For value betting hands, we can take the derivative of the expression for $EV(x)$ with respect to $x$ and show that it is always positive. We consider the max and min betting hands first:

    \begin{align*}
        s = U \implies \frac{d}{dx} EV(x) & = 2U + 1 > 0  \\
        s = L \implies \frac{d}{dx} EV(x) & = 2L + 1 > 0
    \end{align*}

    For intermediate value betting hands, we can use the chain rule to show that the EV is increasing in $x$:

    \begin{align*}
        \frac{d}{dx} EV(x) & = \frac{\partial EV(x)}{\partial x} + \frac{dEV(x)}{ds} \frac{d s}{d x} \\
    \end{align*}

    From the value optimality condition (equation \ref{eq:valueoptimality} in the main text), we know that $\frac{dEV(x)}{ds} = 0$ (otherwise, the bettor could gain value by varying the bet size). Clearly, $\frac{\partial EV(x)}{\partial x} > 0$.  Therefore, the derivative is positive, and the EV is increasing in $x$ for all value betting hands.
\end{customproof}

\subsection{Discussion: Strong Hands and Risk-Reward Tradeoffs}

It is worth noting that the bettor's strongest hands (right edge) actually seem to become less likely to make any profit more than the ante as limits increase. These strongest hands make very large bets, which force all but the strongest hands to fold, but win huge pots when they do get called.

In more complicated poker variants, it is common to "slowplay" strong hands by checking or making small bets to induce bluffs from the opponent. In LCP, there is only one betting round and the caller is not allowed to raise, both of which make slowplaying obsolete. With extremely strong hands, the benefit of winning a large pot when betting big outweighs the lower likelihood of getting called.

This strategic pattern demonstrates a fundamental tension in poker: extracting maximum value from strong hands requires finding the optimal balance between bet size (which determines pot size when called) and calling frequency (which decreases as bet size increases). In LCP's Nash equilibrium, the strongest hands resolve this tension by accepting a lower calling frequency in exchange for winning much larger pots.

\end{document}
