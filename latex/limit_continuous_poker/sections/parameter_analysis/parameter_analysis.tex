\documentclass[../../main/main.tex]{subfiles}
\begin{document}

\section{Parameter Analysis}

\subsection{Effect of Increasing $U$}

\subsubsection{Expected Payoff of Value-Betting Hands}

It may seem unsurprising that strong hangs become less likely to get called as limits increase, but what about the actual expected payoff of these hands? Does the expected value of each hand continue increasing as we increase $U$? The answer is no. In fact, for any fixed hand strength $x$, the expected payoff of that hand increases in $U$ only for large enough $x$ or small enough $U$, after which it decreases. This feels counterintuitive; increasing $U$ only gives the bettor more options, so how is it possible that the expected payoff of a wide range of their hands decreases? And which hands are gaining expected payoff to offset this? This is a surprising result, and it is worth exploring in more detail.

Recall from section \ref{bettor_ev} that $EV(x)$ denotes the expected payoff of a value-betting hand $x$ in the unique admissible Nash equilibrium.

\begin{theorem}
    \label{thm:payoff_increasing}
    For any value-betting hand strength $x$ and any $L, U$, $EV(x)$ is decreasing in $U$ for all $x$ which make intermediate-sized bets and weak hands which bet the maximum, and increasing in $U$ for strong hands which bet the maximum, above a certain threshold. Specifically, $\frac{d}{dU} EV(x) < 0$ if 
    
    $$x < \max\left(v(U), \frac{1}{2(1+U)} \left( U \frac{\partial x_2}{\partial U} + \frac{U^2 + x_2(1 + 2U)}{1+U} \right) \right)$$

    and $\frac{d}{dU} EV(x) > 0$ otherwise.
\end{theorem}

Before proving the theorem, we will walk through some lemmas which explore how all the relevant variables change as we increase $U$, including the bluffing threshold $x_2$, the bet size $v^{-1}(x)$, and the calling cutoff $c(s)$.

\subsubsection{Bluffing Threshold}

We begin by showing that $x_2$, the boundary hand strength between bluffing and checking, is increasing in $U$. This means that for fixed $L$, increasing the upper limit $U$ makes the bettor bluff with more hands. 

\begin{lemma}
    \label{lem:x2_increasing}
    For any fixed $L, U$,
    $$ \frac{\partial x_2}{\partial U} > 0 $$
\end{lemma}

\begin{customproof}
    Taking the partial derivative of $x_2$ with respect to $U$ and rearranging terms gives:
    
    $$\frac{\partial x_2}{\partial U} = \frac{18 (L+1)^6 (U+1)^2}{\left(A_1 + L^3 A_2 + 3 L^2 A_1 + 3 L A_1 \right)^2}$$

    Which is always positive since $L \in [0, U]$, $U \in [0, \infty)$, and $A_1, A_2$ are both positive-coefficient polynomials in $L$ and $U$.   
\end{customproof}

\subsubsection{Bet Size}

We now show that if we fix $x$ at any intermediate value-betting hand strength (betting neither the minimum nor maximum bet size) and then increase $U$, the bet size $s$ made by $x$ decreases. The intermediate value-betting hands are exactly $x \in [x_3, v(U)]$ and their bet sizes are given by $s = v^{-1}(x)$, so we get the following lemma:

\begin{lemma}
    \label{lem:v_inverse_decreasing}
    For any fixed $x \in [x_3, v(U)]$, 
    \[ 
        \frac{d}{dU} v^{-1}(x) < 0
    \]
\end{lemma}

\begin{customproof}
    Recall that 
    $$v^{-1}(x) = -\frac{\sqrt{(4 x-4) (2 x_2-2)}}{4 x-4}-1$$
    Where $x_2$ is a function of $L$ and $U$. Importantly, $v^{-1}(x)$ is only dependent on $U$ through $x_2$, so we can use the chain rule to take the derivative with respect to $U$:
    \begin{align*}
        \frac{d}{dU} v^{-1}(x) & = \frac{\partial v^{-1}(x)}{\partial x_2} \frac{\partial x_2}{\partial U}
    \end{align*}

    The first term is 

\begin{align*}
    \frac{\partial v^{-1}(x)}{\partial x_2} & = - \frac{1}{\sqrt{(4 x-4) (2 x_2-2)}} \\
    &= - \frac{1}{(v^{-1}(x)+1)(4-4x)}
\end{align*}

    Which is always negative since $x \in [0, 1]$ and $v^{-1}(x) >0 $. We know that the second term is positive by Lemma \ref{lem:x2_increasing}.

    Therefore, the product of the two terms is always negative, so the bet size of intermediate bets is decreasing in $U$.
\end{customproof}

\subsubsection{Calling Cutoff}

Recall that $c(s)$ is defined as the minimum hand strength $y$ which should call a bet of size $s$ and is given in Nash equilibrium by:

$$c(s) = \frac{x_2 + s}{s+1}$$

We are specifically interested in how $c(v^{-1}(x))$ varies with $U$ for $x \in [x_3, v(U)]$, since this represents how the calling cutoff changes both directly from a strategic change, as well as indirectly due to the lower bet size $s$. It turns out that the calling cutoff is increasing in $U$ for all $x \in [x_3, v(U)]$. This is surprising because we just showed that the bet size $s$ is decreasing in $U$, and we expect smaller bets to be called more often. For reasons we will see later, this effect is overpowered by a strategic shift for the caller, who calls less often for all bet sizes as $U$ increases. 

\begin{lemma}
    \label{lem:c_increasing}
    For any fixed $x \in [x_3, v(U)]$, 
    \[ 
        \frac{d}{dU} c(v^{-1}(x)) > 0
    \]
\end{lemma}

\begin{customproof}
    As mentioned above, $c(s)$ is dependent on $U$ in two distinct ways -  directly through $x_2$, which can be interpreted as the caller changing strategy as the game changes - but also indirectly in response to how the bet size $s = v^{-1}(x)$ is dependent on $U$. We use the multivariate chain rule to express $\frac{d}{dU} c(v^{-1}(x))$ in terms of these two dependencies:
    \begin{align*}
        \frac{d}{dU} c(v^{-1}(x)) & = \frac{\partial c(s)}{\partial s} \frac{d v^{-1}(x)}{d U} + \frac{\partial c(s)}{\partial x_2} \frac{\partial x_2}{\partial U}
    \end{align*}
    
    The two partial derivatives of $c(s)$ are:

    $$ \frac{\partial c(s)}{\partial s} = \frac{1-x_2}{(s+1)^2} \; \; \; \text{and} \; \; \; \frac{\partial c(s)}{\partial x_2} = \frac{1}{s+1} $$

    Substituting $s = v^{-1}(x)$, we can further simplify the first term:

    \begin{align*}
        \frac{\partial c(s)}{\partial s} \bigg|_{s=v^{-1}(x)} & = \frac{1-x_2}{(v^{-1}(x)+1)^2} \\
        & = 2-2x
    \end{align*}

    And as we showed in the proof of Lemma \ref{lem:x2_increasing}, $\frac{d v^{-1}(x)}{d U}$ is given by:

    $$ \frac{d v^{-1}(x)}{d U} = \frac{-1}{(v^{-1}(x)+1)(4-4x)} \frac{\partial x_2}{\partial U} $$

    Substituting all of these with $s = v^{-1}(x)$:

    \begin{align*}
        \frac{d}{dU} c(v^{-1}(x)) & = 
        \left(2-2x\right) 
        \left(\frac{-1}{(v^{-1}(x)+1)(4-4x)}\right) 
        \left(\frac{\partial x_2}{\partial U}\right) 
        + \left(\frac{1}{v^{-1}(x)+1}\right) 
        \left(\frac{\partial x_2}{\partial U}\right)\\
        & = \left( \frac{1}{v^{-1}(x)+1} \right) 
        \left( \frac{\partial x_2}{\partial U} \right)
        \left( -\frac{2-2x}{4-4x} + 1\right) \\
        & = \left( \frac{1}{v^{-1}(x)+1} \right) 
        \left( \frac{\partial x_2}{\partial U} \right)
        \left( \frac{1}{2} \right) \\
    \end{align*}

    The first term is positive since $v^{-1}(x) > 0$ and the second term is positive by Lemma \ref{lem:x2_increasing}. Therefore, the product of the two terms is positive, so the calling cutoff is increasing in $U$ for all $x \in [x_3, v(U)]$.

\end{customproof}

\subsubsection{Proof of Theorem \ref{thm:payoff_increasing}}

Having these tools, we can now finally return to the proof of Theorem \ref{thm:payoff_increasing}.

\begin{customproof}
    Recall the expected payoff of a value-betting hand $x$:

    \begin{align*}
        EV(x) & = \frac{1}{2} c(s) + (x - c(s)) \left(s+\frac{1}{2}\right) + (1-x) \left(-s-\frac{1}{2}\right)
    \end{align*}

    We break the proof into two cases:

    \textbf{Case 1 ($x > v(U)$):} In this case, hand $x$ bets the maximum amount $U$. The expected value is:
    
    $$ EV(x) = \frac{1}{2} c(U) + (x - c(U)) \left(U+\frac{1}{2}\right) + (1-x) \left(-U-\frac{1}{2}\right) $$ 

    and the derivative is:

    \begin{align*}
        \frac{d}{dU} EV(x) & = \frac{\partial EV(x)}{\partial s} \bigg|_{s=U} + \frac{\partial EV(x)}{\partial c(s)} \left( \frac{\partial c(s)}{\partial s}\bigg|_{s=U} + \frac{\partial c(s)}{\partial x_2} \frac{\partial x_2}{\partial U} \right) \\
        & = \left( \frac{\partial EV(x)}{\partial s} + \frac{\partial EV(x)}{\partial c(s)} \frac{\partial c(s)}{\partial s}\right)\bigg|_{s=U}   +  \left( \frac{\partial EV(x)}{\partial c(s)} \frac{\partial c(s)}{\partial x_2} \frac{\partial x_2}{\partial U} \right) \\
        & = \frac{dEV(x)}{ds}\bigg|_{s=U}   +  \left( \frac{\partial EV(x)}{\partial c(s)} \frac{\partial c(s)}{\partial x_2} \frac{\partial x_2}{\partial U} \right) \\
        & = \frac{dEV(x)}{ds}\bigg|_{s=U}  - \frac{U}{1+U} \cdot \frac{\partial x_2} {\partial U}
    \end{align*} 

    We want to know exactly when the above expression is positive. $\frac{\partial x_2}{\partial U}$ is always positive by Lemma \ref{lem:x2_increasing}, and is independent of $x$. At $x = v(U)$, we know that $\frac{dEV(x)}{ds} = 0$ from \ref{valueoptimality}, so at $x = v(U)$, the entire expression is negative and $EV(x)$ is decreasing in $U$. What about for larger $x$? We can expand the expression further. The partial derivatives of $EV(x)$ are:

    \begin{align*}
        \frac{\partial EV(x)}{\partial s} & = 2x - 1 - c(s) \\
        \frac{\partial EV(x)}{\partial c(s)} & = - s
    \end{align*}

    Plugging these in and rearranging terms, we can say that $EV(x)$ is increasing in $U$ if 

    \begin{align*}
        x & > \frac{1}{2(1+U)} \left( U \frac{\partial x_2}{\partial U} + \frac{U^2 + x_2(1 + 2U)}{1+U} \right)
    \end{align*}

    Where nothing on the right hand side is dependent on $x$. This means that for any fixed $L, U$, this gives a threshold value for $x$ below which $EV(x)$ is decreasing in $U$, and above which it is increasing in $U$.

    \textbf{Case 2 ($x_3 < x < v(U)$):} In this case, hand $x$ makes an intermediate-sized bet $s = v^{-1}(x)$. There are two distinct factors influencing the derivative $\frac{d}{dU} EV(x)$, namely the change in bet size $s = v^{-1}(x)$ and the change in calling cutoff $c(v^{-1}(x))$. By the multivariate chain rule, we can express the derivative as:

    \begin{align*}
        \frac{d}{dU} EV(x) & = \frac{\partial EV(x)}{\partial s} \frac{d v^{-1}(x)}{d U} + \frac{\partial EV(x)}{\partial c(s)} \frac{d c(v^{-1}(x))}{\partial U}
    \end{align*}

    The partial derivatives of $EV(x)$ are:

    \begin{align*}
        \frac{\partial EV(x)}{\partial s} & = 2x - 1 - c(s) \\
        \frac{\partial EV(x)}{\partial c(s)} & = - s
    \end{align*}

    The second is clearly negative. We can verify that the first must be positive if we go back to the constraints which gave us the Nash equilibrium. For the bet size $v^{-1}(x)$ to be optimal, we required that 

    $$ -s \frac{\partial c(s)}{\partial s} - c(s) + 2v(s) - 1 = 0$$

    Or equivalently, if we substitute $s= v^{-1}(x)$ and $v(s) = x$ and rearrange:

    $$ 2x -1 -c(v^{-1}(x)) = v^{-1}(x) \frac{\partial c(s)}{\partial s} > 0$$

    Since $\frac{\partial c(s)}{\partial s} = \frac{1-x_2}{(s+1)^2} > 0$, and $s$ is positive by definition.

    We know from Lemma \ref{lem:v_inverse_decreasing} that $\frac{d v^{-1}(x)}{d U} < 0$ and from Lemma \ref{lem:c_increasing} that $\frac{d c(v^{-1}(x))}{\partial U} > 0$.

    Combining everything, we see that both terms in $\frac{d}{dU} EV(x)$ are products of negative and positive, making both terms negative. Therefore, the expected payoff is decreasing in $U$ for all $x \in [x_3, v(U)]$.

\end{customproof}


\end{document}